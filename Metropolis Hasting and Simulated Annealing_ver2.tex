%% LyX 2.1.2 created this file.  For more info, see http://www.lyx.org/.
%% Do not edit unless you really know what you are doing.
\documentclass[oneside,reqno]{amsart}
\usepackage[T1]{fontenc}
\usepackage[utf8]{inputenc}
\pagestyle{plain}
\usepackage{mathrsfs}
\usepackage{amsthm}
\usepackage{amstext}
\usepackage{amssymb}
\usepackage{esint}
\usepackage[numbers]{natbib}
\usepackage[unicode=true,pdfusetitle,
 bookmarks=true,bookmarksnumbered=false,bookmarksopen=false,
 breaklinks=false,pdfborder={0 0 1},backref=section,colorlinks=false]
 {hyperref}

\makeatletter
%%%%%%%%%%%%%%%%%%%%%%%%%%%%%% Textclass specific LaTeX commands.
\numberwithin{equation}{section}
\numberwithin{figure}{section}

%%%%%%%%%%%%%%%%%%%%%%%%%%%%%% User specified LaTeX commands.
%2multibyte Version: 5.50.0.2960 CodePage: 1250
%*********************************  Cut Here **********************************************
%% Latex File
%Title:  	
%Bruce K. Driver						(619) 458-0544	(Home)	
%University of CA, San Diego			(619) 534-2648 (Work)
%Department of Mathematics				(619) 534-5273  (Fax)
%9500 Gilman Drive, Dept. 0112			Email: driver@euclid.ucsd.edu
%La Jolla, CA 92093-0112				http://math.ucsd.edu/~driver/
% **********************************************************************************************
%\usepackage[notcite]{showkeys}
%\usepackage{graphicx}
%\usepackage{amscd}
%\usepackage{a4}
%unnumbered theorem environments
%\iffalse
%\usepackage[notcite]{showkeys}
%\usepackage{babel}
%\usepackage{babel}

\usepackage{amsthm}
\usepackage{mathrsfs}
%\usepackage[notcite]{showkeys}
\usepackage{amsthm}
\usepackage{datetime}
\usepackage{mathrsfs}
\usepackage{amsthm}
\usepackage{amsfonts}
\usepackage{comment}
\usepackage{graphicx}\setcounter{MaxMatrixCols}{30}
%TCIDATA{OutputFilter=latex2.dll}
%TCIDATA{Version=5.50.0.2960}
%TCIDATA{Codepage=1250}
%TCIDATA{CSTFile=amsart.cst}
%TCIDATA{Created=Wednesday, August 10, 2011 10:30:47}
%TCIDATA{LastRevised=Tuesday, November 03, 2015 14:08:24}
%TCIDATA{<META NAME="GraphicsSave" CONTENT="32">}
%TCIDATA{<META NAME="SaveForMode" CONTENT="1">}
%TCIDATA{BibliographyScheme=BibTeX}
%TCIDATA{<META NAME="DocumentShell" CONTENT="Articles\SW\bruce_plain">}
%TCIDATA{Language=American English}
%BeginMSIPreambleData
\providecommand{\U}[1]{\protect\rule{.1in}{.1in}}
%EndMSIPreambleData


\numberwithin{equation}{section}
\numberwithin{figure}{section}
\numberwithin{equation}{section}
\numberwithin{figure}{section}
\numberwithin{equation}{section}
\numberwithin{figure}{section}
\iffalse
\oddsidemargin= -0.2in
\evensidemargin= -0.2in
\textheight= 9in
\topmargin= -0.2in
\textwidth= 6.9in
\fi
\theoremstyle{plain}
\newtheorem{theorem}{Theorem}[section]
\newtheorem{corollary}[theorem]{Corollary}
\newtheorem{lemma}[theorem]{Lemma}
\newtheorem{convention}[theorem]{Convention}
\newtheorem{mtheorem}[theorem]{Meta-Theorem}
\newtheorem{proposition}[theorem]{Proposition}
\newtheorem{htheorem}[theorem]{Heuristic Theorem}
\newtheorem{axiom}{Axiom}
\newtheorem{solution}[theorem]{Solution}
\newtheorem{summary}[theorem]{Summary}
\renewenvironment{proof}[1][Proof]{\textbf{#1.} }{\ \rule{0.5em}{0.5em}}
\theoremstyle{definition}
\newtheorem{definition}[theorem]{Definition}
\newtheorem{comments}[theorem]{Comments}
\newtheorem{notation}[theorem]{Notation}
\newtheorem{notations}[theorem]{Notations}
\newtheorem{example}[theorem]{Example}
\theoremstyle{remark}
\newtheorem{remark}[theorem]{Remark}
\newtheorem{remarks}[theorem]{Remarks}
\newtheorem{fact}[theorem]{Fact}
\theoremstyle{plain}
\newtheorem{conjecture}[]{Conjecture}
\newtheorem{claim}[]{Claim}
\newtheorem{computation}[]{Computation}
\theoremstyle{definition}
\newtheorem{assumption}[]{Assumption}
\newtheorem{exercise}{Exercise}[section]
\numberwithin{equation}{section}
\newcommand*{\bigchi}{\mbox{\Large$\chi$}}
\excludecomment{comments}

\makeatother

\begin{document}

\date{September 6, 2017}


\title{Metropolis Hasting And Simulated Annealing}

\maketitle
\tableofcontents{}

%\today\ \emph{File:\jobname{.tex}}

%\setcounter{tocdepth}{1}

\setcounter{section}{-1}

%\newpage


\section{Goal}

This note is a draft of a brief introduction of Metropolis-Hasting
Algorithm and stimulated annealing. Reader should read this note with
caution. There are many references about these two topics, e.g. \citep{Geyer}
and \citep{Lee}


\section{Metropolis-Hasting Algorithm\label{sec.2}}

It is well known that there is a curse of dimensionality when we sample
states following a probability distributions $\pi$ in a high dimensional
sample space of states $S$. Metropolis-Hasting is one of popular
approaches to solve this issue. The main idea to make Metropolis-Hasting
Algorithm work is to find a transition kernel density $p(x,y)$ on
$S\times S$ such that the the Markov chain induced by the transition
kernel density $p(x,y)$ has the limiting probability $\pi_{L}=\pi.$
If we can find such a transition kernel density, then we can start
the Markov chain with a random state $s_{0}$ and let the Markov chain
proceed. Then, we can start to collect states induced by the Markov
chain after sufficiently many iterations in the chain and the states
we collected follow the limiting probability distribution $\pi_{L}=\pi.$

The key to implement Metropolis-Hasting Algorithm successfully is
to find another transition kernel density $q(\cdot,\cdot)$ on $S\times S$
such that it follows the following properties: 
\begin{enumerate}
\item Given $x\in S,$it is easy to sample states from $q(x,\cdot).$ 
\item Let $f$ be a density of a probability distribution $\pi$ with respect
to a probability measure $\mu$ on $S$, i.e. $\pi(d\mu)=fd\mu$.
For all $x\in S$, the support of $q(x,\cdot)$ contains the support
of $f(\cdot)$, i.e. if $f(y)>0$, then $q(x,y)>0$. {[}Actually,
we can weaken this condition in practice.{]} 
\end{enumerate}
States $\{X_{t}\}$ is denoted to be a Markov chain in the Metropolis-Hasting
Algorithm. The following is Metropolis-Hasting Algorithm to samples
states in $S$ following a probability distribution $\pi$: 
\begin{enumerate}
\item For $t=0$, pick an initial state $X_{0}=s_{0}\in S$ such that $f(s_{0})>0$
and assign the last step $T$ to be a large positive number. 
\item Generate a candidate $Y$ follows from the probability distribution
$q(X_{t},\cdot)$ 
\item Generate a random number $\beta$ following a uniform distribution
between $0$ and $1$ 
\item Update the state $X_{t+1}$as $\begin{cases}
Y & \textrm{if }\alpha(X_{t},Y)>\beta\\
X_{t} & \textrm{otherwise}
\end{cases}$ where 
\begin{equation}
\alpha(x,y):=min\left\{ \frac{f(y)q\left(y,x\right)}{f(x)q\left(x,y\right)},1\right\} .\label{eq:2.2}
\end{equation}

\item $t\leftarrow t+1$ and go to Step 2 until $t>T.$ 
\end{enumerate}
Metropolis-Hasting Algorithm builds a Markov chain $\left\{ X_{t}\right\} $.
Then we pick $T_{eqm}<T$ large enough such that all states generated
after $T_{eqm}$ follow very closely to a probability distribution
$\pi$ and we collect states $\left\{ X_{t}\right\} _{T_{eqm}<t<T}$
as a sample on $S$ following probability distribution $\pi.$

\begin{remark} There are three remarks when we implement Metropolis-Hasting
Algorithm in practice.
\begin{enumerate}
\item Both probability distributions $q(x,\cdot)d\mu(\cdot)$ and $\pi(\cdot)$
are not necessarily to be normalized, that we do not require $\int_{y\in S}q(x,y)d\mu(y)$
and $\pi(S)=\int_{y\in S}f(y)dy$ equals to $1.$ Especially the normalizing
constant to normalize $q(x,\cdot)$ or $\pi(\cdot)$ are unknown in
some cases. 
\item The transition kernel density $q(x,y)$ is not necessarily to be symmetric,
i.e. $q(x,y)=q(y,x).$ 
\item If the initial state $X_{0}$ we start at has $f(X_{0})=0,$ then
the update in Step 4 becomes invalid as the denominator of Eq. \ref{eq:2.2}
becomes zero.
\end{enumerate}
\end{remark}

INFORMALLY, the transition kernel density in the Markov chain of the
Metropolis-Hasting Algorithm can be written as 

\begin{equation}
p(x,y)=q(x,y)\alpha(x,y)+r(x)\delta_{x}(y)\label{eq:2.1}
\end{equation}
where $\delta_{x}(y)dy$ is a Dirac measure, i.e.

\[
\int_{y\in A}f(y)\delta_{x}(y)dy=\begin{cases}
f(x) & \textrm{if }x\in A\\
0 & \textrm{otherwise}
\end{cases}
\]


and 
\[
r(x)=1-\int_{y\in S}q(x,y)\alpha(x,y)d\mu(y).
\]


The first term on R.H.S. in Eq. \ref{eq:2.1} stands for the probability
of all accepted move from a state $x$ to a state $y$ $(X_{t+1}$is
updated as $Y$) while the second term on R.H.S. in Eq. \ref{eq:2.1}
stands for the probability of all unaccepted move from a state $x$
to a state $y$ $(X_{t+1}$ is updated as $X_{t}$).


\section{Simulated Annealing \label{sec.3}}

One application of Metropolis-Hasting Algorithm is to search for an
optimal point( global minimum/ global maximum) point of a function.
The function needs not to be continuous. Suppose $S$ is a sample
spaces of states (it can be a high dimensional space) and $E:S\rightarrow\mathbb{R}$
is an objective function. Let's say we search for the global minimum
point of $E.$ The main idea of simulated annealing is to find a probability
distribution $\pi$ on $S$ such that $\pi$ is extremely large at
the global minimum point and almost zero elsewhere. Then, we apply
Metropolis-Hasting to generate a Markov chain whose states follows
the probability distribution $\pi$ at the very end. In other words,
the last state in the Markov chain is very close to the global minimum
point. One common example to construct such a probability distribution
$\pi$ is 
\begin{equation}
f(x)=\exp\left(-E\left(x\right)\right).\label{eq.3.1}
\end{equation}
where $f$ is the density function corresponding to $\pi$, i.e. $\pi(d\mu)=fd\mu(x).$
However, in order to increase the ability to get rid of local minimum
point and stabilize the convergence of states in the Markov chain,
we also introduce another parameter, temperature $\mathscr{T}$ to
Eq. \ref{eq.3.1} and the probability distribution becomes 
\[
\pi(x)=\exp(-\frac{E(x)}{\mathscr{T}})
\]
where $\mathscr{T}$ decreases slowly along steps in the Markov chain.

Let $\left\{ X_{t}\right\} $ be a Markov chain in simulated annealing
algorithm for minimizing $E:S\rightarrow\mathbb{R}.$ The simulated
annealing algorithm is as follows: 
\begin{enumerate}
\item For $t=0$, pick an initial state $X_{0}=s_{0}\in S$ and initialize
both the last step $T$ and the temperature $\mathscr{\mathscr{T}=T_{0}}$
to be a large positive number. 
\item Generate a candidate $Y$ follows from the probability distribution
$q(X_{t},\cdot)$ 
\item Generate a random number $\beta$ following a uniform distribution
between $0$ and $1$ 
\item Update the state $X_{t+1}$as $\begin{cases}
Y & \textrm{if }\alpha(X_{t},Y)>\beta\\
X_{t} & \textrm{otherwise}
\end{cases}$ where 
\[
\alpha(x,y):=min\left\{ \exp(-\frac{E(y)-E(x)}{\mathscr{T}})\frac{q\left(y,x\right)}{q\left(x,y\right)},1\right\} .
\]

\item $t\leftarrow t+1$ and $\mathscr{T}\leftarrow\mathscr{T_{0}}(0.5^{\frac{t}{1000}})$
and go to Step 2 until $t>T.$ 
\end{enumerate}
The last state $X_{T}$ should be very close to the global minimum
point of the function $E.$ There is a remark of the simulated annealing
algorithm.

\begin{remark}There are two remarks about the simulated annealing
algorithm
\begin{enumerate}
\item In the step 2 of the algorithm, without further information, we have
no ideas which direction a state should move so it can get closer
to the global minimum point. Therefore, all proposed directions in
the step 2 should be equally likely. Hence, the transition kernel
$q$ should be symmetric, i.e.$q$$\left(x,y\right)=q\left(y,x\right).$
Common examples of such a $q(x,y)$ is a Gaussian distribution, random
walk or a uniform distribution. Furthermore, if $q(\cdot,\cdot)$
is symmetric, then $\alpha(x,y)=min\left\{ \exp(-\frac{E(y)-E(x)}{\mathscr{T}}),1\right\} $.
\item The update formular of temperature $\mathscr{T}$ in the step 5 means
$\mathscr{T}$ decreases by half in every $1000$ iteration.
\end{enumerate}
\end{remark} 

\bibliographystyle{plainnat}
\bibliography{monteCarlo}

\end{document}
